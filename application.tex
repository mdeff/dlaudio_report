\part{Application} \label{part:application}
% Apply the developed algorithm to a problem, MGR in our case.

\chapter{Problem} \label{chap:problem}
% Explain the problem we aim at solving: recognizing musical genre while being robust for e.g. deployement on a smartphone which may well be in a noisy environment.

\gls{MGR} is a common task for \gls{MIR} and is the usual task for the yearly \gls{MIREX}\footnote{\url{http://www.music-ir.org/mirex/wiki/MIREX_HOME}}. The \gls{MGR} problem is the task of automatically recognizing the musical genre of an unknown audio clip given a set of labeled clips. A clip is unknown in the sense that only the raw audio is available, there is no access to any meta-data.

%\section{Performance evaluation}
The accuracy of the classified clips is used as a proxy for the discriminative power of the learned representations.
%While the classification accuracy does not say all about \gls{MGR}, it proved useful for comparison purposes and is easy to asses.

\section{Dataset}

The system's performance is to be evaluated on GTZAN\footnote{Available at \url{http://marsyasweb.appspot.com/download/data\_sets/}.}, the most-used public dataset for evaluation in machine listening research for \gls{MGR} \cite{sturm2014survey} which was created by Tzanetakis and Cook for their work in the domain \cite{tzanetakis2002GTZAN}. It consists of 1000 30-second audio clips (all truncated to $660000$ samples to not bias the classifier in any way) with 100 examples in each of 10 different categories: blues, classical, country, disco, hiphop, jazz, metal, pop, reggae and rock. All clips are sampled at 22050 Hz.

% nuit-blanche.blogspot.ch/2013/12/the-curious-case-of-gtzan-dataset.html
The composition and integrity of the dataset has been analyzed in \cite{sturm2012GTZANanalysis}. It is known that the dataset contains recurrent faults: repetitions, mislabelings, and distortions. These faults obviously challenge the interpretability of any result derived using it. \cite{sturm2013GTZANcritic} goes even further and disprove the claims that all \gls{MGR} systems are affected in the same ways by these faults, and that the performances of \gls{MGR} systems in GTZAN, working with the same data and faults, are still meaningfully comparable.
%Our results are however only compared to themselves, making these critics essentially irrelevant.

%\section{Genres}
% How we can qualitatively distinguish genres.

% Music theory
% Explain how music is constructed.
%\section{Notes and frequencies}
%\section{Harmonics, chords and harmonies}

%In our genre recognition setting, a patch spectrogram\footnote{We will define it precisely in \chapref{model}, think of it as the spectrogram of some short time frame.} $\x \in \R^n$ is embedded in an $n$-dimensional space. Not all $n$-dimensional signals, however, are plausible spectrograms. We may think of the set of plausible spectrograms to lie on a lower $k$-dimensional manifold which is embedded in the $n$-dimensional space.


%%%%%%%%%%%%%%%%%%%%%%%%%%%%%%%%%%%%%%%%%%%%%%%%%%%%%%%%%%%%%%%%%%%%%%%%%%%%%%%


\chapter{System} \label{chap:system}
% Visual diagram: pre-processing, features extraction, post-processing, classification, voting
% Inspyred by [ref lecun] which uses sparse auto-encoders.

The design of the genre recognition system is inspired by \cite{lecun2011PSDaudio}, which uses a layer of sparse auto-encoder to learn sparse representations of audio spectrograms targeted at \gls{MGR}. Although they call their encoder technique \gls{PSD} \cite{lecun2010PSD}, it is effectively a sparse auto-encoder \cite{bengio2009learningDeepAI}. The structured auto-encoder developed in \partref{algorithm} generalizes their loss function as an energy function and introduces a structuring term (\secref{manifold_learning}).

The system mainly consists of three independent building blocs:
\begin{enumerate}
	\item \textbf{Preprocessing}, whose goal is to transform the raw audio signal into a time-frequency representation. It is essentially a first feature extraction pass, built on prior knowledge about signal processing and audio signals in particular. The spectrograms are further sliced in short time frames to provide time invariance.
	\item \textbf{Feature extraction}: the set of spectrogram slices is given as input vectors to the proposed structured auto-encoder who transforms them into a sparse and structured representation.
	\item \textbf{Classification}: the extracted features are used for genre classification after a post-processing step analog to feature pooling. The accuracy is assessed by a cross-validation scheme.
\end{enumerate}

{\color{red} system figure}

\section{Preprocessing} \label{sec:preprocessing}
% Frames, CQT, LCN, features / sample scaling.
% A LCN may further increase the accuracy [ref LeCun].

\paragraph{Frames.}
Each clip is divided into short frames of $n_a = 1024$ samples, which roughly corresponds to 46 ms of raw audio, in order to provide translation invariance. A 50\% overlap between consecutive frames introduces redundancy in the data. The GTZAN dataset is thus decomposed in $N=1288000$ frames of dimensionality $n_a = 1024$.

\paragraph{\gls{CQT}.}
% Spectrogram with geometrically spaced frequencies.
A spectral representation of each of those frames is computed via the \gls{CQT} with $n_s=96$ filters spanning four octaves from C\textsubscript{2} to C\textsubscript{6} at quarter-tone resolution, where C\textsubscript{4} is the middle C in the scientific pitch notation \cite{young1939ScientificPitch}. The A440 tuning standard sets A\textsubscript{4} = 440 Hz \cite{A440std}.

Apart from the constant quality factor, i.e. a constant frequency over bandwidth ratio, an important property of the \gls{CQT} is that the center frequencies of the filters are logarithmically spaced, so that consecutive notes in the musical scale are linearly spaced.
This transform is generally well suited to musical data: (i) the logarithm scale requires fewer frequency bins to cover a given range effectively, which proves useful when frequencies span several octaves. (ii) It mirrors the human auditory system, whereby at lower frequencies spectral resolution is better, whereas temporal resolution improves at higher frequencies. (iii) The harmonics of musical notes form a pattern characteristic of the timbre of the instrument: as the fundamental frequency changes, the relative position of these harmonics remains constant.

This step is an intelligent dimensionality reduction from $n_a = 1024$ to $n_s = 96$. Intelligent in the sense that it extracts useful features, as demonstrated by the baseline experiment (\secref{spectrograms}).

The implementation uses \textit{librosa}\footnote{Available at \url{https://github.com/bmcfee/librosa/}.}, a Python library which implements an efficient algorithm proposed by the authors of \cite{schorkhuber2010CQTtoolbox}. An efficient and accurate implementation of the necessary sample rate conversion is provided by \textit{libsamplerate}\footnote{Available at \url{http://www.mega-nerd.com/SRC/index.html}.}.

\paragraph{\gls{LCN}.}
A subtractive and divisive \gls{LCN}, described in \cite{lecun2010LCN}, may then be applied to the spectrograms in order to enhance the contrast.
Consider a point in the spectrogram and its neighborhood along both the time and frequency axes weighted by a Gaussian window. First, the average of the weighted neighborhood is subtracted from each point (the subtractive part). Then, each point is divided by the standard deviation of its new weighted neighborhood (the divisive part).

The technique enforces competition between neighboring points in the spectrogram, so that low-energy signals are amplified while high-energy ones are muted. The entire process can be seen as a simple form of \gls{AGC}. It tries to inverse the heat diffusion, similarly to a shock filter \cite{osher1990shockFilters}. The idea of contrast normalization is inspired by visual neuroscience models \cite{lyu2008LCNneuro1, pinto2008LCNneuro2}.

\paragraph{Scaling.}
The range of the independent variables is finally rescaled in $[0,1]$ to eliminate any bias toward features who have a broad range of values. Each feature then contributes approximately proportionately to the distance between two features vectors. Another commonly used scaling method is standardization: the independent variables are rescaled to have zero-mean and unit-variance. Yet another method is to rescale each features vector to unit length, which usually means dividing each component by the Euclidean length of the vector.

\section{Feature extraction}
% Most accurate extraction with original model.
% Much faster approximations by ignoring some terms of the objective.

In a transductive learning paradigm \cite{vapnik1998transductiveLearning}, the auto-encoder model (\chapref{model}) is trained on the entire dataset (which includes the training and test sets without labels). Under this paradigm, test samples are known in advance, and the system is simply asked to produce labels for them. This paradigm was chosen for speed reasons: since training is computationally expensive, it is best done only once per experiment. The training phase is essentially the application of the algorithm presented in \chapref{optimization} to solve \eqnref{model} over the whole dataset. A structured and sparse representation is readily available for each spectrogram of the dataset, they are a by-product of model fitting.

Note that while that paradigm is used for this application, the algorithm proposed in \partref{algorithm} is still purely unsupervised, i.e. it makes no use of the training labels.
Note also that while the system was designed for transductive learning, it can act in a supervised learning way. A predictive model was built during training, and, using the learned parameters, a representation for a previously unknown spectrogram may be inferred with \eqnref{z_exact} or \eqnref{z_direct}.

\section{Classification}

\paragraph{Aggregate features.}
% Analog to feature pooling.
% Goal: change of time scale.
Aggregate features are computed for each song by summing up the frame-level features over 5-second time windows overlapping by half, which has been found to substantially improve classification performance \cite{bergstra2006aggregateFeatures, hamel2010aggregateFeatures}. Since each sparse code records which dictionary elements are present in a given \gls{CQT} frame, these aggregate features vectors can be thought of as histograms recording the number of occurrences of each dictionary element in the time window.

When constructing the graph, the implementation uses the FLANN library \cite{muja2009flann} for a fast approximate k-nearest neighbors search.

\paragraph{\gls{SVM}.}
Aggregate features are then classified by a \gls{SVM} \cite{cortes1995SVM}, which is a non-probabilistic binary linear classifier. It was chosen because it is fast to train and scale well to large datasets (thanks to the few support vectors), which is an important consideration in \gls{MIR}.

In a one-vs-the-rest strategy to multi-class classification, \gls{SVM} basically constructs a set of maximum-margin hyperplanes which separate a class from all the others. The alternative approach to multi-class, one-vs-one, requires to train one classifier per pair of class.

The implementation uses the \textit{scikit-learn} Python framework \cite{sklearn} which provides wrappers to \textit{libsvm} \cite{libsvm} and \textit{liblinear} \cite{liblinear}, two independent implementations of the \gls{SVM}. We observed no significant variations between the two.

\paragraph{Majority voting}
% Voting further gives a cheap 
The genre prediction for a clip is given by a majority vote between the 12 aggregate features. Instead of using it in a winner-take-all fashion, this information could be exploited as a confidence level about the chosen class.


%%%%%%%%%%%%%%%%%%%%%%%%%%%%%%%%%%%%%%%%%%%%%%%%%%%%%%%%%%%%%%%%%%%%%%%%%%%%%%%


%\chapter{Implementation} \label{chap:implementation}

%\section{Framework}
% Stack: numpy, scipy, matplotlib, scikit-learn, librosa
% Tools: IPython notebook, CDK cluster, matplotlib, h5py, librosa
% Explain why and how data is stored (layout) via HDF5.

%\subsection{PyUNLocBoX}
% PyUNLocBoX: explains how it works

%\section{Design}

%\section{Performance} \label{sec:performance}

%\subsection{Algorithm}
% FISTA vs PD implementation

%\subsection{Approximate KNN search}
% How FLANN works, what are the alternatives.
% Techniques: KDtree, ball, local hashes (LHS)
% cosine to euclidean

%\subsection{Optimization for space}
% Optimization for space: avoid copy, modify in place, float32, store Z as scipy.sparse

%\subsection{Optimization for speed}
% Optimization for speed: ATLAS/OpenBLAS, float32 (memory bandwidth), projection in the ball (not on the sphere)
% ATLAS mono-threaded (at least on Ubuntu), OpenBLAS multi-threaded.
% Linear algebra: optimized version of BLAS: ATLAS and OpenBLAS. (LINPACK)


%%%%%%%%%%%%%%%%%%%%%%%%%%%%%%%%%%%%%%%%%%%%%%%%%%%%%%%%%%%%%%%%%%%%%%%%%%%%%%%


\chapter{Results} \label{chap:results}
% Results and comparisons
% Include discussion in results ?

% All the simulations on a reduced dataset because of computation time.
% Only one simulation on the full set for comparison with others. So without noise, with graph.
The iterative optimization scheme presented in \chapref{optimization} is, by definition, rather computationally heavy. Despite having been sped up by an order of magnitude already (see \secref{performance}), it is still quite sub-optimal from a computational point-of-view. As the algorithm itself is still a prototype, we did not want to invest too much time to further optimize its implementation. For this reason, the following experiments where all conducted on a subset of the dataset.

Furthermore, the classification accuracy was measured on individual frames, i.e. before majority voting, rather than on whole clip. The rational is to capture the confidence about the class of a clip. Another positive impact is the reduction of the variance caused by the increased number of samples.

Note that no contrast enhancement (the \gls{LCN} described in \secref{preprocessing}) was applied to the spectrograms for the following experiments.

Note that the provided results (\chapref{results}) have been  without this step.

The first presented result shows that the sparse and structured representation extracted from the spectrograms by the structured auto-encoder defined in \partref{algorithm} is indeed more discriminative than the spectrograms in a classification task.

The second experiment shows that structured representations are robust to noise.

All the conducted experiments, whether they succeed or failed, may be viewed online\footnote{The IPython notebooks are stored on GitHub and can be visualized at \url{http://nbviewer.ipython.org/github/mdeff/dlaudio_results/}.}.

All the results were obtained via git revision *** of the code, available on GitHub and revision *** of the PyUNlocBoX\footnote{Available at}.

% State what the hyper-parameters are.

\paragraph{Cross-validation}
All the results therein are averages of 20 runs of 10-fold cross-validation.

Following standard practice,
classification performance was measured by 10-fold cross-
validation. For each fold, 100 songs were randomly selected
to serve as a test set, with the remaining 900 serving as train-
ing data. This procedure was repeated 20 times, and the re-
sults averaged to produce a final classification accuracy.

\section{Spectrograms} \label{sec:spectrograms}
% Show example spectrograms of jazz / blues. Show how they are different (music theory) and can such be distinguished.

% Or whole dataset.
The spectrograms are indeed discriminative features. On an experiment to classify $N = $ frames, we observed a classification accuracy of $00$, whereas classification with raw audio yields an accuracy of $00$. Classifications with raw spectrogram is our baseline for further experiments: we will report improvements of the extracted features with respect to the baseline.

%\subsection{Baseline} \label{sec:baseline}
%We first applied linear SVM (or other) to raw audio, \gls{CQT} spectrogram and \gls{LCN} normalized spectrogram. \tabref{tab:comparison} indeed shows that each transformation, i.e. extraction of higher level features, of the raw audio makes sense and improve the classification accuracy.

\section{Learned features}
% Show some atoms: harmonics, chords, harmonies, drums.
% Probably not on full CQT, should be on octaves.
%Via music theory, we know that music is constructed via some building blocks, e.g.

Although we have not observed meaningful atoms (yet), \cite{lecun2010PSD} showed that when dictionaries where trained on individual octaves, they discovered harmonics, chords and harmonies without any prior about music theory.

% LeCun.
One can see single notes and what appear to be series of linearly
spaced notes, which could correspond to chords, harmonics
or harmonies. Note that some of the basis functions appear
to be inverted, since the code coefficients can be negative.
A number of the learned basis functions also seem to have
little recognizable structure.


\section{Convergence analysis}

The encoder fidelity sub-objective is 2 orders of magnitude lower than the other sub-objectives, which suggests that the addition of the encoder does not alter the features extracted via the complete scheme defined by \eqnref{model}.

\section{Descriptive features vectors}
% Show aggregated features vectors for some genres.
% How is it qualitatively more discriminative than the spectrogram ?

\section{Hyper-parameters tuning}
% Test matrix for hyper-parameters on smaller problem (i.e. less frames).
% For ld, le, lg, m
% Numerical parameters: Nouter (enough when no more inner), rtol
% Graph parameters: K, Csigma, kernel?, metric
% Classification: C, Nvectors
% Pre-processing: na=1024, ns=96

Overcompleteness, defined by the hyper-parameter $m$, must be evaluated by considering the number of code units and the effective dimensionality of the input as given by \gls{PCA}.

\section{Classification accuracy}
% On the whole dataset, not very good in comparison with others.
% Measured via 10-fold cross-validation
% How classification is improved by features learning, introducing the encoder / graph.
% Comparison with other techniques.

M: However in the noiseless environment the addition of the graph is not significant.
X: oui, mais tu ne dois pas oublier que tu fais du *transductive* learning, c'est a dire tu apprends les features en utilisant training + TEST data. C'est pour cela que tu n'as pas une amelioration significative. Si on faisait du *supervised* learning, c'est a dire on utilise seulement les TRAINING data, alors ce probleme est bcp plus challenging que le transductive probleme, et la je pense que nos resultats avec graph seraient bien meilleures! C'est un commentaire que je te conseille d'ajouter a ton resultat pour le mettre en perspective. Aussi la premiere chose a faire apres le PDM est de faire du *supervised* learning avec graph et le comparer avec no graph, je pense que l'on aura des (bonnes) surprises!


\section{Scarce training set}
% Il faut juste garder à l'esprit que le graphe (et le dictionnaire) est appris sur le dataset complet. Ça peut cependant être utile si on a beaucoup de données à disposition mais que les labels coûtent chers.

\section{Robustness to noisy data}
% Final experiement: baseline, graph-less, graph vs noise level

The complete simulation report may be visualized at (or downloaded from GitHub.)

\section{Discussion}
% Is the model appropriate for the problem ?

\begin{table}
	\begin{center}
		\begin{tabular}{|l|l|l|}
			\hline
			Classifier & Features & Acc. (\%) \\
			\hline
			linear SVM & raw audio & 0 \\
			linear SVM & CQT spectrogram & 0 \\
			linear SVM & normalized CQT spectrogram & 0 \\
			linear SVM & features from \eqnref{basispursuit} & 0 \\
			linear SVM & features from \eqnref{minz} & 0 \\
			linear SVM & features from \eqnref{extraction} & 0 \\
			\hline
		\end{tabular}
	\end{center}
	\caption{Genre recognition accuracy of various algorithms on GTZAN.}
	\label{tab:accuracy_comparison}
\end{table}

While our results did not attain the state-of-the-art (yet) in \gls{MGR} on GTZAN, depicted in \tabref{accuracy_comparison};
we demonstrated the usefulness of an important property of the proposed model: the conservation of the structure in the data, which allows the system to be robust to noisy data as well as being able to generalize with a very scarce training set.
%we made the point that our model is useful and that we shall continue to research on it.
Higher classification accuracies are probably achievable by fine-tuning the hyper-parameters and introducing further tricks: e.g. by applying a \gls{LCN} to the \gls{CQT} spectrograms or working on individual octaves, two techniques used by \cite{lecun2011PSDaudio}. We may even further improve the performance of our model by creating a better graph, i.e. a graph more adapted to the problem at hand, by fine-tuning the hyper-parameters.