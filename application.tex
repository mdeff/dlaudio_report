\part{Application} \label{part:application}
% Apply the developed algorithm to a problem, MGR in our case.

\chapter{Problem} \label{chap:problem}
% Explain the problem we aim at solving: recognizing musical genre while being robust for e.g. deployement on a smartphone which may well be in a noisy environment.
We think that this problem satisfies the hypothesizes we made in 

%\section{Genres}
% How we can qualitatively distinguish genres.

% Music theory
% Explain how music is constructed.
%\section{Notes and frequencies}
%\section{Harmonics, chords and harmonies}

%In our genre recognition setting, a patch spectrogram\footnote{We will define it precisely in \chapref{model}, think of it as the spectrogram of some short time frame.} $\x \in \R^n$ is embedded in an $n$-dimensional space. Not all $n$-dimensional signals, however, are plausible spectrograms. We may think of the set of plausible spectrograms to lie on a lower $k$-dimensional manifold which is embedded in the $n$-dimensional space.


%%%%%%%%%%%%%%%%%%%%%%%%%%%%%%%%%%%%%%%%%%%%%%%%%%%%%%%%%%%%%%%%%%%%%%%%%%%%%%%


\chapter{System} \label{chap:system}
% Visual diagram: pre-processing, feature extraction, post-processing, classification, voting
% Inspyred by [ref lecun] which uses sparse auto-encoders.

The design of the system is inspired by \cite{lecun2011PSDaudio} which uses sparse auto-encoders for the same task.

Similarly to us, the authors of \cite{lecun2011PSDaudio} used a layer of sparse auto-encoder to learn sparse representations of audio spectrograms. They introduced a technique called \gls{PSD} to quickly infer an approximation of the sparse codes \cite{lecun2010PSD}, which is effectively a sparse auto-encoder.

The structuring auto-encoder developed in \partref{algorithm} generalizes their loss function as an energy function and introduces a structuring term. We also introduced a well-posed iterative scheme to solve the resulting non-convex optimization problem in \chapref{learning}.

\section{Pre-processing}
% Frames, CQT, LCN, feature / sample normalization
% A LCN may further increase the accuracy [ref LeCun].

\subsection{Frames}

\subsection{Constant-Q transform}
% Spectrogram with geometrically spaced frequencies.

\section{Feature extraction}
% Most accurate extraction with original model.
% Much faster approximations by ignoring some terms of the objective.

\section{Classification}
% Feature aggregation (feature vectors), linear SVM, majority voting (winner take all)
% Voting further gives a cheap confidence level about our choice.
% Multi-class: one-vs-one / one-vs-the-rest

\subsection{Feature aggregation}

\subsection{Support Vector Machine}
% Small introduction to SVM.

\subsection{Majority voting}

\subsection{Cross-validation}


%%%%%%%%%%%%%%%%%%%%%%%%%%%%%%%%%%%%%%%%%%%%%%%%%%%%%%%%%%%%%%%%%%%%%%%%%%%%%%%


\chapter{Implementation} \label{chap:implementation}

\section{Framework}
% Stack: numpy, scipy, matplotlib, scikit-learn, librosa
% Tools: IPython notebook, CDK cluster, matplotlib, h5py, librosa
% Explain why and how data is stored (layout) via HDF5.

\subsection{PyUNLocBoX}
% PyUNLocBoX: explains how it works

\section{Design}

\section{Performance}

\subsection{Algorithm}
% FISTA vs PD implementation

\subsection{Approximate KNN search}
% How FLANN works, what are the alternatives.
% Techniques: KDtree, ball, local hashes (LHS)
% cosine to euclidean

\subsection{Optimization for space}
% Optimization for space: avoid copy, modify in place, float32, store Z as scipy.sparse

\subsection{Optimization for speed}
% Optimization for speed: ATLAS/OpenBLAS, float32 (memory bandwidth), projection in the ball (not on the sphere)
% ATLAS mono-threaded (at least on Ubuntu), OpenBLAS multi-threaded.
% Linear algebra: optimized version of BLAS: ATLAS and OpenBLAS. (LINPACK)


%%%%%%%%%%%%%%%%%%%%%%%%%%%%%%%%%%%%%%%%%%%%%%%%%%%%%%%%%%%%%%%%%%%%%%%%%%%%%%%


\chapter{Results} \label{chap:results}
% Results and comparisons
% Include discussion in results ?

\section{Spectrograms}
% Show example spectrograms of jazz / blues. Show how they are different (music theory) and can such be distinguished.

\section{Learned features}
% Show some atoms: harmonics, chords, harmonies, drums.
% Probably not on full CQT, should be on octaves.
%Via music theory, we know that music is constructed via some building blocks, e.g.

Although we have not observed meaningful atoms (yet), \cite{lecun2010PSD} showed that when dictionaries where trained on individual octaves, they discovered harmonics, chords and harmonies without any prior about music theory.

\section{Descriptive feature vectors}
% Show aggregated feature vectors for some genres.
% How is it qualitatively more discriminative than the spectrogram ?

\section{Hyper-parameters tuning}
% Test matrix for hyper-parameters on smaller problem (i.e. less frames).
% For ld, le, lg, m
% Numerical parameters: Nouter (enough when no more inner), rtol
% Graph parameters: K, Csigma, kernel?, metric
% Classification: C, Nvectors
% Pre-processing: na=1024, ns=96

Overcompleteness, defined by the hyper-parameter $m$, must be evaluated by considering the number of code units and the effective dimensionality of the input as given by \gls{PCA}.

\section{Classification accuracy}
% Measured via 10-fold cross-validation
% How classification is improved by feature learning, introducing the encoder / graph.
% Comparison with other techniques.

\section{Robustness to noisy data}
% Final experiement: baseline, graph-less, graph vs noise level

\section{Discussion}
% Is the model appropriate for the problem ?

While higher classification accuracies are probably achievable by fine-tuning the hyper-parameters and introducing further tricks; e.g. by applying a \gls{LCN} to the \gls{CQT} spectrograms or working on individual octaves, two techniques used by \cite{lecun2011PSDaudio}; we made the point that our model is useful and that we shall continue to research on it. We may even further improve the performance of our model by creating a better graph, i.e. a graph more adapted to the problem at hand, by fine-tuning the hyper-parameters.